% !TEX options=--shell-escape
\title{First Response Essay:\\
\large{The Production of Information
in an Online World}}
\date{\today}

\documentclass[12pt]{article}
\usepackage[utf8]{inputenc}
\usepackage[parfill]{parskip}
\usepackage{amsmath}
%\usepackage[outputdir=/home/pascal/.config/sublime-text-3/Cache]{minted}

\usepackage{blindtext}
\usepackage[most]{tcolorbox} 
\definecolor{block-gray}{gray}{0.95}


\newtcolorbox{zitat}[2][]{%
    colback=block-gray,
    grow to right by=-10mm,
    grow to left by=-10mm, 
    boxrule=0pt,
    boxsep=0pt,
    breakable,
    enhanced jigsaw,
    borderline west={4pt}{0pt}{gray},
    title={#2\par},
    colbacktitle={block-gray},
    coltitle={black},
    fonttitle={\large\bfseries},
    attach title to upper={},
    #1,
}

\begin{document}
\maketitle

\section{Research Question \& Motivation}
In their paper \textit{The Production of INformation in an Online World} the
authors Julia Cagé, Nicolas Hervé and Marie-Luce Viaud research the following
question:

\begin{zitat}{Research Question}
  Given the limited intellectual property protection in news media,
  what is the extent of copying in online news production, and what are the
  incentives to produce original content?
\end{zitat}

While their motivation is complex, we can boild it down to two main concerns:
\begin{enumerate}
  \item Influence of bad journalism on the political landscape.
  \item How to safeguard the quality of journalism.
\end{enumerate}

Both those concern come from the point of view, that press and news media are
the so called fourth estate which basically is the believe, that press and
news media have a responsibility of providing quality journalism to properly
inform the voters. Furthermore, they argue that current copyright laws are too
weak and need to be stronger, to prevent an "abuse of power" from the fourth
estate.

\section{Contributions}
Comparing to previous work, the authors have to following contributions:
\begin{itemize}
  \item Data: Construction of a new, very detailed and large data set.
  \item Methodology: Application of new and state-of-the-art NLP algorithm.
  \item Methodology: Creation of a new plagiarism algorithm.
\end{itemize}

Furthermore, they state that their work also complements a growing empirical
literature on copyright.

\section{Methods \& Data}
To be able to tackle the given research question, the authors want to record
the spread of a news story. They do that by constructing a dataset that
consists of the main French news media for the whole of the year 2013. This
includes newspapers, television channels, radio stations, pure online media and
news agencies.

To be able to track a news story, one has to be able to detect a news story.
To be able to achieve that, the authors develop a topic detection algorithm
that identifies each news event. They basically do that by creating a class
for each news topic and then just go back on the time axis.

The first story to break a news event is considered the original story. They
then use their plagiarism algorithm to detect copies whereas they don't
consider stories based on articles from news agencies copies, since the purpose
of a news agency is to give a news event to media outlets.

To measure the impact of a news story, they track statistics like shares and
likes on Facebook as well as Twitter.

\section{Issue, Comments \& Improvements}
\paragraph{Impact Measure} First I want to point out, that it is hard to judge
if their impact measure is actually worth while. The problem is, that they can
only view public information yet the fact that it is public might be a bias
in itself. It might be, that people have an agenda when sharing a news article
publicly. For example, if I read a good article, I rather share it directly
i.e. privately with someone rather than posting it publicly but if I wanted to
generate awareness for a specific topic, I'd post it publicly.

I also want to add, that there are studies showing that most of the articles
that are being shared online aren't being read. The only thing that was read,
was the title. Reinforcing my argument about an agenda based sharing desicion.

\paragraph{New Structrues} Since this is a response essay
and not an academic work, I'd like to think about some more controversial
topics.

To me, it seems that the authors have a very
strong bias towards the classical order of society resp. the structure of
the fourth estate. The fourth estate has always been and will always be abused
to lobby voters. So the current structure of article based news might simply
not be a good model to distribute news. Working towards stronger copyright
laws might harden this structure and in turn might keep us locked to the old
way.

It might be, that we are better off by letting the news be as free as possible.
Free meaning that there won't be a lot of copyright protection in place. This
might allow for new technologies and structure to emerge like a personlized
news assistant. A chat bot that derlivers the news to you. Such a tool could
be checked for biases much easier than a normal news paper.

While it is good to think about how to preserve the status-quo, I also think
it might be worthwhile to propose a very different solution. Maybe newspaper
simply don't work in the digital age because the way young people consume
news media might change drastically. Instead of reading an article, they
probably much rather watch a YouTueb video about a selected topic or maybe
even watch a TikTok.

To reiterate: Of course, the point of the paper wasn't to investigate
alternative news distribution models but rather investigate the current one.
But in a next step they do use their results to make a conclusion and that's
where the bias takes place. I think that the conclusion could have been a good
opportunity to take a broader stance.

\paragraph{Disregard of Youth} Their impact measure relies on public
statistics from Facebook and Twitter. This might disregard the youth because
most young people don't use Facebook and it's questionable how many use
Twitter. Furthermore, their way of informing themselves might not be by reading
an article, but by watching a video on YouTube or even TikTok.

While their dataset and impact measure might fit together, both seem to
disregard the youth.


\end{document}
